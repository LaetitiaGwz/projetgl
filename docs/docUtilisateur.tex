\documentclass[a4,12pt]{article}

\usepackage[francais]{babel}
\usepackage[utf8]{inputenc}
\usepackage[T1]{fontenc}
\usepackage[babel=true]{csquotes}
\usepackage{amsmath}
\usepackage{amssymb}
\usepackage{float}
\usepackage{graphicx}
\usepackage{wrapfig}
\usepackage{hyperref}
\usepackage{array,multirow,makecell}
\frenchbsetup{StandardLists=true}
\usepackage{enumitem}
\setlength\parindent{20pt}
\begin{document}
\begin{titlepage}
\title{ Documentation Utilisateur du compilateur Ddca}
\author{Gonthier Florentin, Bergeron Matthieu, Beaupère Matthias,\\ Fischman Adrien, Geoffroy Germain}
\date{}

\maketitle

\rule[0.5ex]{\textwidth}{0.2mm}
Cette documentation destinée à un utilisateur du compilateur deca facilite la prise en main du compilateur.
Elle explique les différentes limites imposées au langage soumit au compilateur ainsi que la signification des
différentes erreurs de compilation et d'exécution. Elle propose enfin des exemples d'utilisation en exposant
le code assembleur généré par le compilateur.

\rule[0.5ex]{\textwidth}{0.2mm}

\end{titlepage}
\tableofcontents
\newpage

\section{Limitations du compilateur}

\section{Limitations de la bibliothèque math}
La classe Math contient différentes méthodes :
\begin{itemize}
    \item float ulp(float f)
    \item float sin(float f)
    \item float cos(float f)
    \item float asin(float f)
    \item float atan(float f)
\end{itemize}
Chaque méthode a été testée grâce aux valeurs retournées par les fonctions prédéfinies de Java.Math. Ces méthodes sont des approximations de valeurs mathématiques, dont on mesure la précision grâce aux ULP (Unit in the Last Place).\\
Pour une analyse complète et détaillée des algorithmes et de leurs précisions respectives, il faut se reporter à la documentation de la classe Math. \\
On présente ici, sous forme de tableau, les différents résultats obtenus.
\subsection{Méthode ulp.}
La méthode ulp donne le résultat exacte sans approximation.\\
\\

\hspace{-2cm}
\begin{tabular}{|c|c|c|c|c|c|}

\hline 
 & intervalle & nb erreur($>1 ulp$) & erreur max(ulp) & pas & nb tests \\
\hline 
ulp & $[0; 1\ 000]$ & 0 & $0$ & $2^{-14}$ & 16 384 000\\
\hline 
ulp & $[10\ 000; 100\ 000]$ & 0 & $0$ & $2^{-7}$ & 11 520 000\\
\hline 
ulp & $[2^{30}; 2^{31}]$ & 0 & $0$ & $128$ & 8 388 608\\
\hline
\end{tabular}

\subsection{Méthode sin.}
La méthode sin est implémenté grâce au développement en série entière de sinus et d'une réduction de Cody and Waite sur [-$\pi /8; \pi /8$].\\
La méthode sin donne une approximation à 5 ulp d'écart au maximum.\\
Pour des petits angles ($<\pi /8$), la précision est de 1 ulp dans le pire cas.\\
\\

\hspace{-4cm}\begin{tabular}{|c|c|c|c|c|c|c|}

\hline 
 & intervalle & nb erreur($>1 ulp$) & erreur max(ulp)& erreur moyenne(ulp) & pas & nb tests \\
\hline 
sin & $[0; \pi /8]$             & 0       & $1$ & null        &$2^{-23}$ & 3 294 199\\
\hline 
sin & $[\pi /8; 2\pi]$          & 1117838 & $5$ &$2.13$ & $2^{-20}$ & 6 176 623\\
\hline 
sin & $[1\ 000; 1\ 000 + 2\pi]$     & 17777    & $5$ &$2.13$ & $2^{-14}$ & 102 944\\
\hline 
sin & $[100\ 000; 100\ 000 + 2\pi]$ & 127      & $4$ &$ 2.20$ & $2^{-7}$ & 804\\
\hline
\end{tabular}

\subsection{Méthode cos.}
La méthode cos est implémenté grâce au développement en série entière de cosinus et d'une réduction de Cody and Waite sur [-$\pi /8;\pi /8$].\\
La méthode cos donne une approximation à 6 ulp d'écart au maximum.\\
Pour des petits angles ($<\pi /8$), la précision est de 2 ulp dans le pire cas.\\
\\

\hspace{-4cm}\begin{tabular}{|c|c|c|c|c|c|c|}

\hline 
 & intervalle & nb erreur($>1 ulp$) & erreur max(ulp)& erreur moyenne(ulp) & pas & nb tests \\
\hline 
cos & $[0; \pi /8]$             & 365      & $2$ & 2.0        &$2^{-23}$ & 3 294 199\\
\hline 
cos & $[\pi /8; 2\pi]$          & 1335192 & $6$ &$2.18$ & $2^{-20}$ & 6 176 623\\
\hline 
cos & $[1\ 000; 1\ 000 + 2\pi]$     & 20987    & $5$ &$2.19$ & $2^{-14}$ & 102 944\\
\hline 
cos & $[100\ 000; 100\ 000 + 2\pi]$ & 162     & $4$ &$ 2.12$ & $2^{-7}$ & 804\\
\hline
\end{tabular}

\subsection{Méthode asin.}
La méthode asin est implémenté grâce au développement en série entière de la fonction arc sin.\\
Par imparité de la fonction arcsinus, on ne donne le résultat que pour des nombres positifs.\\
La méthode asin donne une approximation à 4 ulp d'écart au maximum.\\
\\

\hspace{-3cm}
\begin{tabular}{|c|c|c|c|c|c|c|}
\hline
 & intervalle & nb erreur($>1 ulp$) & erreur moyenne & erreur max(ulp) & pas & nb tests \\
\hline
asin & $[0;0.5]$ & 0 & 0 & 1 &$2^{-20}$ & 524288\\
\hline
asin & $[0.5;0.72]$ & 1974 & 2 & 2 & $2^{-20}$ & 230687\\
\hline
asin & $[0.72;0.98]$ & 22660 & 2.12 & 4 & $2^{-20}$ & 272630\\
\hline
asin & $[0.98;1]$ & 1687 & 2.43 & 7 & $2^{-20}$ & 20972\\
\hline
\end{tabular}

\subsection{Méthode atan.}
La méthode atan est implémenté grâce aux polynômes de Hermite.\\
Par imparité de la fonction arctangente, on ne donne le résultat que pour des nombres positifs.\\
La méthode atan donne une approximation à 2 ulp d'écart au maximum.\\
\\

\hspace{-3cm}
\begin{tabular}{|c|c|c|c|c|c|c|}

\hline
 & intervalle & nb erreur($>1 ulp$) & erreur moyenne & erreur max(ulp) & pas & nb tests \\
\hline
atan & $[0; 0.6875]$ & 0 & 0 & 1 &$2^{-20}$ & 720 896\\
\hline
atan & $[0.6875; 1.1875]$ & 0 & 0 & 1 & $2^{-20}$ & 524 288\\
\hline
atan & $[0.1875; 2]$ & 43 006 & 2.0 & 2.0 & $2^{-23}$ & 6 815 744\\
\hline
atan & $[2; 100]$ & 0 & 0 & 1.0 & $2^{-12}$ & 401 408\\
\hline
atan & $[100; 10\ 000]$ & 0 & 0 & 1.0 & $2^{-7}$ & 1 267 200\\
\hline
\end{tabular}
\section{Recensement des erreurs}

\section{Exemples}

\end{document}