\documentclass[a4,12pt]{article}

\usepackage[francais]{babel}
\usepackage[utf8]{inputenc}
\usepackage[T1]{fontenc}
\usepackage[babel=true]{csquotes}
\usepackage{amsmath}
\usepackage{amssymb}
\usepackage{float}
\usepackage{graphicx}
\usepackage{wrapfig}
\usepackage{hyperref}
\usepackage{array,multirow,makecell}
\frenchbsetup{StandardLists=true}
\usepackage{enumitem}
\setlength\parindent{20pt}
\usepackage[left=1cm,right=1cm,top=2cm,bottom=2cm]{geometry}
\begin{document}
\begin{titlepage}
\title{ Documentation Utilisateur du compilateur Ddca}
\author{Gonthier Florentin, Bergeron Matthieu, Beaupère Matthias,\\ Fischman Adrien, Geoffroy Germain}
\date{}

\maketitle

\rule[0.5ex]{\textwidth}{0.2mm}
Cette documentation destinée à un utilisateur du compilateur deca facilite la prise en main du compilateur.
Elle explique les différentes limites imposées au langage soumit au compilateur ainsi que la signification des
différentes erreurs de compilation et d'exécution. Elle propose enfin des exemples d'utilisation en exposant
le code assembleur généré par le compilateur.

\rule[0.5ex]{\textwidth}{0.2mm}

\end{titlepage}
\tableofcontents
\newpage

\section{Limitations du compilateur}

\section{Limitations de la bibliothèque maths}

\section{Recensement des erreurs}

\subsection{Erreur de lexicographie}
\begin{tabular}{|l|l|l|}
\hline
   Description de l'erreur & Exemple & Résultat \\
   \hline
   blabla & int a = 5.3/0.0 & Lexer error...  \\
   \hline
\end{tabular}
\subsection{Erreur de syntaxe hors-contexte}
\begin{tabular}{|l|l|l|}
\hline
   Description de l'erreur & Exemple & Résultat \\
   \hline
   blabla & int a = 5.3/0.0 & Syntax error  \\
   \hline
\end{tabular}
\subsection{Erreur de syntaxe contextuelle}
\begin{tabular}{|l|l|l|}
\hline
   Description de l'erreur & Exemple & Résultat \\
   \hline
   blabla & int a = 5.3/0.0 & Context error...  \\
   \hline
\end{tabular}
\subsection{Erreur à l'éxécution}

\begin{tabular}{|l|l|l|}
\hline
   Description de l'erreur & Exemple & Résultat \\
   \hline
   Débordement arithmétique & int a = 5.4/0.0; & Error :  \\
      &  & overflow during arithmetic operation \\
   \hline
   Débordement du tas & Trop d'appels à new() & Error : heap overflow \\
   \hline
   Débordement de la pile & Trop d'appel& Error : stack overflow \\
         & de fonctions récursives &\\
   
   \hline
   Déréférencement d'un pointeur NULL & Object a = null;  & Error :  \\
   & a.equals(a); & dereferencing null pointer \\
   \hline
\end{tabular}



\section{Exemples}
\subsection{Hello World}

\textbf{Code deca :} 
\begin{verbatim}
	{
        println("Hello");
	}
\end{verbatim}
\textbf{Code assembleur :} 
\begin{verbatim}
; start main program
	TSTO #4
	BOV stack_overflow
	ADDSP #4
	LOAD #null, R0
	STORE R0, 1(GB)
	LOAD code.Object.equals, R0
	STORE R0, 2(GB)
; Main program
; Beginning of main instructions:
	WSTR "Hello"
	WNL
	HALT
init.Object:
	RTS
code.Object.equals:
	TSTO #2
	BOV stack_overflow
	PUSH R2
	PUSH R3
	LOAD -2(LB), R2
	LOAD -3(LB), R3
	CMP R2, R3
	SEQ R0
	POP R3
	POP R2
	RTS
; end main program
arith_overflow:
	WSTR "Error : overflow during arithmetic operation"
	ERROR
stack_overflow:
	WSTR "Error : stack overflow"
	ERROR
heap_overflow:
	WSTR "Error : heap overflow"
	ERROR
dereferencement.null:
	WSTR "Error : dereferencing null pointer"
	ERROR
\end{verbatim}

\subsection{Fibonacci itératif}
\textbf{Code deca :} 
\begin{verbatim}
{
    int n = 25;
    int i = 1;
    int first = 0;
    int second = 1;
    int tmp;
    while(i <= n){
        tmp = first + second;
        first = second;
        second = tmp;
        i = i + 1;
    }
    print(first);
}
\end{verbatim}
\textbf{Code assembleur :} 
\begin{verbatim}
; start main program
	TSTO #9
	BOV stack_overflow
	ADDSP #9
	LOAD #null, R0
	STORE R0, 1(GB)
	LOAD code.Object.equals, R0
	STORE R0, 2(GB)
; Main program
	LOAD #25, R2
	STORE R2, 3(GB)
	LOAD #1, R3
	STORE R3, 4(GB)
	LOAD #0, R4
	STORE R4, 5(GB)
	LOAD #1, R5
	STORE R5, 6(GB)
; Beginning of main instructions:
DebutWhile0:
	LOAD 4(GB), R0
	LOAD 3(GB), R1
	CMP R1, R0
	BGT EndWhile0
	LOAD 5(GB), R6
	ADD 6(GB), R6
	STORE R6, 7(GB)
	LOAD 6(GB), R2
	STORE R2, 5(GB)
	LOAD 7(GB), R2
	STORE R2, 6(GB)
	LOAD 4(GB), R2
	ADD #1, R2
	STORE R2, 4(GB)
	BRA DebutWhile0
EndWhile0:
	LOAD 5(GB), R1
	WINT
	HALT
init.Object:
	RTS
code.Object.equals:
	TSTO #2
	BOV stack_overflow
	PUSH R2
	PUSH R3
	LOAD -2(LB), R2
	LOAD -3(LB), R3
	CMP R2, R3
	SEQ R0
	POP R3
	POP R2
	RTS
; end main program
arith_overflow:
	WSTR "Error : overflow during arithmetic operation"
	ERROR
stack_overflow:
	WSTR "Error : stack overflow"
	ERROR
heap_overflow:
	WSTR "Error : heap overflow"
	ERROR
dereferencement.null:
	WSTR "Error : dereferencing null pointer"
	ERROR
\end{verbatim}
\subsection{Grosse expression et option -r 4}
Nous testons ici le bon fonctionnement de l'option -r qui permet de limiter le nombre de registres disponibles. On remarque que l'option fonctionne correctement, puisqu'aucun registre supérieur à R3 n'est utilisé. L'expression est aussi correctement évaluée à 94. \\
\textbf{Code deca :} 
\begin{verbatim}
{
    int a = ( ( (5 + (3 % 1)) - 2 + ((2 - 2 + 3)*5 + 2 - 2) + (1 - 1) - (1/3 + 54) + 2
    - ( 2 + 3*2*3*(9*10/11)) + 1 - 8) % 4
    *
    ( (5 + (3 % 1)) - 2 + ((2 - 2 + 3)*5 + 2 - 2) + (1 - 1) - (1/3 + 54) + 2
    - ( 2 + 3*2*3*(9*10/11)) + 1 - 8) % 4
    -
    ( (5 + (3 % 1)) - 2 + ((2 - 2 + 3)*5 + 2 - 2) + (1 - 1) - (1/3 + 54) + 2
    - ( 2 + 3*2*3*(9*10/11)) + 1 - 8) % 4
    + (2*23*3 - 1)
    ) / ( (23%2) + 2*2*(2-(3*(4-3+1-1))))
    *
     (( (5 + (3 % 1)) - 1 + ((2 - 2 + 3)*5 + 2 - 2) + (1 - 1) - (1/3 + 54) + 2
    - ( 2 + 3*2*3*(9*10/11)) + 1 - 8) % 4) ;
    print(a);
}
\end{verbatim}
\textbf{Code assembleur :} 
\begin{verbatim}
; start main program
	TSTO #5
	BOV stack_overflow
	ADDSP #5
	LOAD #null, R0
	STORE R0, 1(GB)
	LOAD code.Object.equals, R0
	STORE R0, 2(GB)
; Main program
	LOAD #5, R2
	LOAD #3, R3
	REM #1, R3
	BOV arith_overflow
	ADD R3, R2
	SUB #2, R2
	TSTO #1
	BOV stack_overflow
	PUSH R3
	LOAD #2, R3
	SUB #2, R3
	ADD #3, R3
	MUL #5, R3
	ADD #2, R3
	SUB #2, R3
	ADD R3, R2
	POP R3
	TSTO #1
	BOV stack_overflow
	PUSH R3
	LOAD #1, R3
	SUB #1, R3
	ADD R3, R2
	POP R3
	TSTO #1
	BOV stack_overflow
	PUSH R3
	LOAD #1, R3
	QUO #3, R3
	BOV arith_overflow
	ADD #54, R3
	SUB R3, R2
	POP R3
	ADD #2, R2
	TSTO #1
	BOV stack_overflow
	PUSH R3
	LOAD #2, R3
	TSTO #1
	BOV stack_overflow
	PUSH R2
	LOAD #3, R2
	MUL #2, R2
	MUL #3, R2
	TSTO #1
	BOV stack_overflow
	PUSH R3
	LOAD #9, R3
	MUL #10, R3
	QUO #11, R3
	BOV arith_overflow
	MUL R3, R2
	POP R3
	ADD R2, R3
	POP R2
	SUB R3, R2
	POP R3
	ADD #1, R2
	SUB #8, R2
	REM #4, R2
	BOV arith_overflow
	TSTO #1
	BOV stack_overflow
	PUSH R3
	LOAD #5, R3
	TSTO #1
	BOV stack_overflow
	PUSH R2
	LOAD #3, R2
	REM #1, R2
	BOV arith_overflow
	ADD R2, R3
	POP R2
	SUB #2, R3
	TSTO #1
	BOV stack_overflow
	PUSH R2
	LOAD #2, R2
	SUB #2, R2
	ADD #3, R2
	MUL #5, R2
	ADD #2, R2
	SUB #2, R2
	ADD R2, R3
	POP R2
	TSTO #1
	BOV stack_overflow
	PUSH R2
	LOAD #1, R2
	SUB #1, R2
	ADD R2, R3
	POP R2
	TSTO #1
	BOV stack_overflow
	PUSH R2
	LOAD #1, R2
	QUO #3, R2
	BOV arith_overflow
	ADD #54, R2
	SUB R2, R3
	POP R2
	ADD #2, R3
	TSTO #1
	BOV stack_overflow
	PUSH R2
	LOAD #2, R2
	TSTO #1
	BOV stack_overflow
	PUSH R3
	LOAD #3, R3
	MUL #2, R3
	MUL #3, R3
	TSTO #1
	BOV stack_overflow
	PUSH R2
	LOAD #9, R2
	MUL #10, R2
	QUO #11, R2
	BOV arith_overflow
	MUL R2, R3
	POP R2
	ADD R3, R2
	POP R3
	SUB R2, R3
	POP R2
	ADD #1, R3
	SUB #8, R3
	MUL R3, R2
	POP R3
	REM #4, R2
	BOV arith_overflow
	TSTO #1
	BOV stack_overflow
	PUSH R3
	LOAD #5, R3
	TSTO #1
	BOV stack_overflow
	PUSH R2
	LOAD #3, R2
	REM #1, R2
	BOV arith_overflow
	ADD R2, R3
	POP R2
	SUB #2, R3
	TSTO #1
	BOV stack_overflow
	PUSH R2
	LOAD #2, R2
	SUB #2, R2
	ADD #3, R2
	MUL #5, R2
	ADD #2, R2
	SUB #2, R2
	ADD R2, R3
	POP R2
	TSTO #1
	BOV stack_overflow
	PUSH R2
	LOAD #1, R2
	SUB #1, R2
	ADD R2, R3
	POP R2
	TSTO #1
	BOV stack_overflow
	PUSH R2
	LOAD #1, R2
	QUO #3, R2
	BOV arith_overflow
	ADD #54, R2
	SUB R2, R3
	POP R2
	ADD #2, R3
	TSTO #1
	BOV stack_overflow
	PUSH R2
	LOAD #2, R2
	TSTO #1
	BOV stack_overflow
	PUSH R3
	LOAD #3, R3
	MUL #2, R3
	MUL #3, R3
	TSTO #1
	BOV stack_overflow
	PUSH R2
	LOAD #9, R2
	MUL #10, R2
	QUO #11, R2
	BOV arith_overflow
	MUL R2, R3
	POP R2
	ADD R3, R2
	POP R3
	SUB R2, R3
	POP R2
	ADD #1, R3
	SUB #8, R3
	REM #4, R3
	BOV arith_overflow
	SUB R3, R2
	POP R3
	TSTO #1
	BOV stack_overflow
	PUSH R3
	LOAD #2, R3
	MUL #23, R3
	MUL #3, R3
	SUB #1, R3
	ADD R3, R2
	POP R3
	TSTO #1
	BOV stack_overflow
	PUSH R3
	LOAD #23, R3
	REM #2, R3
	BOV arith_overflow
	TSTO #1
	BOV stack_overflow
	PUSH R2
	LOAD #2, R2
	MUL #2, R2
	TSTO #1
	BOV stack_overflow
	PUSH R3
	LOAD #2, R3
	TSTO #1
	BOV stack_overflow
	PUSH R2
	LOAD #3, R2
	TSTO #1
	BOV stack_overflow
	PUSH R3
	LOAD #4, R3
	SUB #3, R3
	ADD #1, R3
	SUB #1, R3
	MUL R3, R2
	POP R3
	SUB R2, R3
	POP R2
	MUL R3, R2
	POP R3
	ADD R2, R3
	POP R2
	QUO R3, R2
	BOV arith_overflow
	POP R3
	TSTO #1
	BOV stack_overflow
	PUSH R3
	LOAD #5, R3
	TSTO #1
	BOV stack_overflow
	PUSH R2
	LOAD #3, R2
	REM #1, R2
	BOV arith_overflow
	ADD R2, R3
	POP R2
	SUB #1, R3
	TSTO #1
	BOV stack_overflow
	PUSH R2
	LOAD #2, R2
	SUB #2, R2
	ADD #3, R2
	MUL #5, R2
	ADD #2, R2
	SUB #2, R2
	ADD R2, R3
	POP R2
	TSTO #1
	BOV stack_overflow
	PUSH R2
	LOAD #1, R2
	SUB #1, R2
	ADD R2, R3
	POP R2
	TSTO #1
	BOV stack_overflow
	PUSH R2
	LOAD #1, R2
	QUO #3, R2
	BOV arith_overflow
	ADD #54, R2
	SUB R2, R3
	POP R2
	ADD #2, R3
	TSTO #1
	BOV stack_overflow
	PUSH R2
	LOAD #2, R2
	TSTO #1
	BOV stack_overflow
	PUSH R3
	LOAD #3, R3
	MUL #2, R3
	MUL #3, R3
	TSTO #1
	BOV stack_overflow
	PUSH R2
	LOAD #9, R2
	MUL #10, R2
	QUO #11, R2
	BOV arith_overflow
	MUL R2, R3
	POP R2
	ADD R3, R2
	POP R3
	SUB R2, R3
	POP R2
	ADD #1, R3
	SUB #8, R3
	REM #4, R3
	BOV arith_overflow
	MUL R3, R2
	POP R3
	STORE R2, 3(GB)
; Beginning of main instructions:
	LOAD 3(GB), R1
	WINT
	HALT
init.Object:
	RTS
code.Object.equals:
	TSTO #2
	BOV stack_overflow
	PUSH R2
	PUSH R3
	LOAD -2(LB), R2
	LOAD -3(LB), R3
	CMP R2, R3
	SEQ R0
	POP R3
	POP R2
	RTS
; end main program
arith_overflow:
	WSTR "Error : overflow during arithmetic operation"
	ERROR
stack_overflow:
	WSTR "Error : stack overflow"
	ERROR
heap_overflow:
	WSTR "Error : heap overflow"
	ERROR
dereferencement.null:
	WSTR "Error : dereferencing null pointer"
	ERROR
\end{verbatim}

\end{document}