\documentclass[a4,12pt]{article}

\usepackage[francais]{babel}
\usepackage[utf8]{inputenc}
\usepackage[T1]{fontenc}
\usepackage[babel=true]{csquotes}
\usepackage{amsmath}
\usepackage{amssymb}
\usepackage{float}
\usepackage{graphicx}
\usepackage{wrapfig}
\usepackage{hyperref}
\usepackage{array,multirow,makecell}
\frenchbsetup{StandardLists=true}
\usepackage{enumitem}
\setlength\parindent{20pt}
\usepackage[left=1cm,right=1cm,top=2cm,bottom=2cm]{geometry}
\begin{document}
\begin{titlepage}
\title{ Documentation Utilisateur du compilateur Ddca}
\author{Gonthier Florentin, Bergeron Matthieu, Beaupère Matthias,\\ Fischman Adrien, Geoffroy Germain}
\date{}

\maketitle

\rule[0.5ex]{\textwidth}{0.2mm}
Cette documentation destinée à un utilisateur du compilateur deca facilite la prise en main du compilateur.
Elle explique les différentes limites imposées au langage soumit au compilateur ainsi que la signification des
différentes erreurs de compilation et d'exécution. Elle propose enfin des exemples d'utilisation en exposant
le code assembleur généré par le compilateur.

\rule[0.5ex]{\textwidth}{0.2mm}

\end{titlepage}
\tableofcontents
\newpage

\section{Limitations du compilateur}

\section{Limitations de la bibliothèque maths}

\section{Recensement des erreurs}

\subsection{Erreur de lexicographie}
\begin{tabular}{|l|l|l|}
\hline
   Description de l'erreur & Exemple & Résultat \\
   \hline
   blabla & int a = 5.3/0.0 & Lexer error...  \\
   \hline
\end{tabular}
\subsection{Erreur de syntaxe hors-contexte}
\begin{tabular}{|l|l|l|}
\hline
   Description de l'erreur & Exemple & Résultat \\
   \hline
   blabla & int a = 5.3/0.0 & Syntax error  \\
   \hline
\end{tabular}
\subsection{Erreur de syntaxe contextuelle}
\begin{tabular}{|l|l|l|}
\hline
   Description de l'erreur & Exemple & Résultat \\
   \hline
   blabla & int a = 5.3/0.0 & Context error...  \\
   \hline
\end{tabular}
\subsection{Erreur à l'éxécution}

\begin{tabular}{|l|l|l|}
\hline
   Description de l'erreur & Exemple & Résultat \\
   \hline
   Débordement arithmétique & int a = 5.4/0.0; & Error :  \\
      &  & overflow during arithmetic operation \\
   \hline
   Débordement du tas & Trop d'appels à new() & Error : heap overflow \\
   \hline
   Débordement de la pile & Trop d'appel& Error : stack overflow \\
         & de fonctions récursives &\\
   
   \hline
   Déréférencement d'un pointeur NULL & Object a = null;  & Error :  \\
   & a.equals(a); & dereferencing null pointer \\
   \hline
\end{tabular}



\section{Exemples}

\end{document}