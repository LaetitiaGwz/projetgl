\documentclass[a4,12pt]{article}

\usepackage[francais]{babel}
\usepackage[utf8]{inputenc}
\usepackage[T1]{fontenc}
\usepackage[babel=true]{csquotes}
\usepackage{amsmath}
\usepackage{amssymb}
\usepackage{float}
\usepackage{graphicx}
\usepackage{wrapfig}
\usepackage{hyperref}
\usepackage{array,multirow,makecell}
\frenchbsetup{StandardLists=true}
\usepackage{enumitem}
\setlength\parindent{20pt}
\begin{document}
\begin{titlepage}
\title{ Documentation Utilisateur du compilateur Ddca}
\author{Gonthier Florentin, Bergeron Matthieu, Beaupère Matthias,\\ Fischman Adrien, Geoffroy Germain}
\date{}

\maketitle

\rule[0.5ex]{\textwidth}{0.2mm}
Cette documentation destinée à un utilisateur du compilateur deca facilite la prise en main du compilateur.
Elle explique les différentes limites imposées au langage soumit au compilateur ainsi que la signification des
différentes erreurs de compilation et d'exécution. Elle propose enfin des exemples d'utilisation en exposant
le code assembleur généré par le compilateur.

\rule[0.5ex]{\textwidth}{0.2mm}

\end{titlepage}
\tableofcontents
\newpage

\section{Limitations du compilateur}

\section{Limitations de la bibliothèque maths}

\section{Recensement des erreurs}

\section{Exemples}

\end{document}